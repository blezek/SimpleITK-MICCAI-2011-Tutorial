\section{Interactive sessions}

%
% The idea I am going with for this section is to introduce a few
% filters. Give the audience a suggested image to load, to experiment
% with the filters as they are being describe. Then pose a problem
% which uses the prior filters for them to solve.
%

\begin{frame}
\frametitle{True utility of SimpleITK}
\begin{itemize}
\item Interacting with data
\end{itemize}
\end{frame}

\subsection{Threshold-based Segmentation}

\begin{frame}[fragile]
\frametitle{Data To Interact With}
\lstpython
\begin{lstlisting}
# Read image, using ipython's tab auto-complete
image = sitk.ReadImage( ``~/SimpleITK-MICCAI-2011-Tutorial/iasem-cells.nrrd'' )

# Get familiar with the image
print image
...
sitk.Show( image )
\end{lstlisting}

This image was obtained with ``Dual-Beam'' or Ion-Abrasion Scanning
Electron Microscope. It has been heavily pre-proccess, and is an X-Z
cross-section of a 3D volume.

\end{frame}

\begin{frame}{Image Masks or Binary Images}

\begin{itemize}
  \item Image Masks are just SimpleITK Images which follow a convention.
  \item Generally, of pixel type uint8\_t with 0-value being background a 1-value being the forgroud.
  \item Masks are used for output of thresholding, binary morphology, etc...
  \item The 1-value was choosen to each of computataion with operators.
  \item If a mask needs to be directly shown, multipy it by 255.
\end{itemize}

\end{frame}



\begin{frame}[fragile]
\frametitle{Threshold-based Segmentation}

\begin{itemize}
  \item {\bf Threshold}\\
    \small
    $ Output(x_i) =
    \begin{cases} Input(x_i) &\text{if $Lower \leq x_i \leq Upper$;}  \\
      OutsideValue            &\text{otherwise.}
    \end{cases} $
    \normalsize
  \item {\bf BinaryThreshold}\\
    \small
    $ Output(x_i) =
    \begin{cases} InsideValue &\text{if $LowerThreshold \leq x_i \leq UpperThreshold$;}  \\
      OutsideValue            &\text{otherwise.}
    \end{cases} $
    \normalsize
  \item {\bf OtsuThreshold} - Automatic Threshold values based on minimizing intra-class variance.
  \item {\bf DoubleThreshold} - A morphology based filters. Uses two sets of thresholds.
\end{itemize}

\lstpython
\begin{lstlisting}
# quick visualizations of masked image
sitk.Show( image * mask )
sitk.Show( .5*image*~mask+image*mask )
\end{lstlisting}

\end{frame}


\begin{frame}{Advanced Geodesic Morphology}

\begin{itemize}
  \item {\bf BinaryOpeningByReconstruction} - Removes binary elements which are smaller than the structuring element.
  \item {\bf BinaryClosingByReconstruction} - Fills binary holes which are smaller then the structuring element.
  \item {\bf BinaryFillHole} - Fills all holes in image.
  \item {\bf BinaryGrindPeak} - Removes all binary elements not connected to boarder.
\end{itemize}


\end{frame}

\begin{frame}{Change Border Problem}

After registration a boarder of the average intensity was added. Change this boarder to 0.

\end{frame}

\begin{frame}{Change Border Solution}
\end{frame}

\subsection{Numpy Interface}
\begin{frame}{SimpleITK with Numpy}
\end{frame}

\begin{frame}{Compute Mode with Numpy Example}
\end{frame}


%
% Begin Section For Hans
%

\subsection{Image Measurements}
\begin{frame}{Image Statistics}
\end{frame}

\begin{frame}{Label Statistics}
\end{frame}


\begin{frame}{Statistics Problem}
\end{frame}

\begin{frame}{Statistics Solution}
\end{frame}


%
% End Section For Hans
%

\subsection{Feature Detection}
\begin{frame}{Edge Detection}
\end{frame}

\begin{frame}{Image Derivatives}
\end{frame}

\begin{frame}{Zero Crossing}
\end{frame}


\begin{frame}{Ridge Detection Problem}
\end{frame}

\begin{frame}{Ridge Detection Solution}
\end{frame}




